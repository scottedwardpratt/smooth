\documentclass[main.tex]{subfiles}
\begin{document}
\setcounter{section}{6}
\section{Template-Based Tutorial}\label{sec:tutorial}

\subsection{Overview}
Once the training points have been generated, the user will input a Real full model based on the given structure, tailored to address their specific problem. The full Real model will input the model\_prior\_info.txt and observable\_info.txt

\subsection{Fake full model templeate}

Within {{\tt local/fakemodel/} there are three files. The first is the information about the model parameters, and their priors is stored in {\tt local/fakemodel/prior\_info.txt}, and information about the observables is store in {\tt local/fakemodel//observable\_info.txt}. The code
{\tt local/fakemodel/templatemod.py} in simulates a fake mathematical model to generate data based on specific functions and coefficients. The user inputs the precise model based on the given structure, tailored to address their specific problem. The code outputs the Model Runs Files, creates the {\tt my\_project/modelruns} directory that stores information for each full-model run.
To run the program, use the command\

\vspace{-20pt}
{\tt
\begin{verbatim} % python3 templatemod.py\end{verbatim}
}


The program reads the parameters, generates obs.txt files as output, and creates output files within the {\tt my\_project/modelruns} directories, The directories
{\tt  my\_project/modelruns/run0/},{\tt  my\_project/modelruns/run1/}, $\cdots$, having files describing the model parameters for each run, along with the output required by the emulator for each specific full-model run. These files contain observables generated by the fake model, including values and associated sigma values. The code's structure includes the `FakeModel` class, which creates random coefficients, calculates specific functions, and prints information related to the fake model. The user should replace the observable files and change the following parameters according to the given Real model as needed.

\end{document}
