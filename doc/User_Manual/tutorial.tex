\documentclass[UserManual.tex]{subfiles}
\begin{document}
\setcounter{section}{7}
\section{Template-Based Tutorial}\label{sec:tutorial}

\subsection{Overview}
A template project directory is provided that the User may copy to their own space, then use this as a foundation from which to embark on their own analysis. This directory includes information files, describing the parameter priors and the observables, that correspond to an artificial model that is also provided as a template. Working through the steps in this section constitutes a tutorial, both for running {\it Simplex Sampler} and for running {\it Smooth Emulator}.

This section describes the steps of how the User would
\begin{enumerate}\itemsep=0pt
\item Copy the required files from the template directory to the User's space, and compile the main programs.
\item Set up the information files describing the priors and observable names.
\item Run {\it Simplex Sampler} to generate the model-parameter values at which the full model will be trained.
\item Run a ``fake'' full model to generate the observables for each of the full-model runs.
\item Tune {\it Smooth Emulator} and write the coefficients to file.
\item Run a program that prompts the User for the coordinates of a point in parameter space, then returns the emulator prediction with its uncertainty.
\end{enumerate}

\subsection{Installation and Compilation}
After completing the necessary prerequisites listed in section \ref{sec:installation}[Installation] and following the steps outlined in section \ref{sec:installation}[Prerequisites] to install the required cmake, eigen, and gsl libraries, and setting the Home Environment Variable by creating the Home Directory as described in section \ref{sec:installation}[Making Home Directory and Setting Home Environment Variable], the user must proceed to clone the smooth and commonutils directories and compile the libraries, as explained in sections \ref{sec:installation}[Downloading] and [Compiling Libraries].

Then, the user can establish a personalized project directory by duplicating the project\_template directory onto their computer. The User should copy the directories {\tt GITHOME\_BAND\_SMOOTH/templates/mylocal} and {\tt GITHOME\_BAND\_SMOOTH/templates/myproject} to a location in their personal space. We will refer to the User's two new directories as {\tt \$\{MY\_LOCAL\}/} and {\tt \$\{MY\_PROJECTS\}/}. For the purpose of this tutorial, the User must compile three main programs. This requires first changing into the {\tt \$\{MY\_LOCAL\}/main\_programs/} directory and entering:\\
{\tt
\begin{verbatim}
   ${MY\_LOCAL}/main_programs% cmake .
   ${MY\_LOCAL}/main_programs% make simplex
   ${MY\_LOCAL}/main_programs% make smoothy_tune
   ${MY\_LOCAL}/main_programs% make smoothy_calcobs
\end{verbatim}
}
The reason these are compiled in the User's space, separate from the main libraries, is that the User may well wish to create their own main programs, and this arrangement allows the User to compile their own versions, while leaving the original programs from the templates directory unchanged. 

For the purpose of the tutorial, there are also some ``fake'' models included in the distribution. For the User's project the fake model, which is very fast numerically, will be replaced by their own numerically intensive model. To compile the fake model used in the tutorial the User should change into the {\tt \$\{MY\_LOCAL\}/main\_programs/} directory and enter:
{\tt
\begin{verbatim}
   ${MY\_LOCAL}/fakemodels% cmake .
   ${MY\_LOCAL}/fakemodels% make fakerhic
\end{verbatim}
}
This particular fake model has six model parameters and six observables, all with names in common use by the RHIC community. The output has absolutely no physical motivation, other than providing some arbitrary functions to emulate. The executable should appear in {\tt \$\{MY\_LOCAL\}/bin/}.

\subsection{Creating Necessary Info Files}
The User will run the software from the {\tt \$\{MY\_PROJECTS\}/} directory. Before a User can run {\it Simplex Sampler} they must create information files that describe the model-parameter priors and list the observable names. Both files are in the {\tt \$\{MY\_PROJECTS\}} directory. The first file is {\tt \$\{MY\_PROJECTS\}/Info/modelpar\_info.txt}. For the purposes of this tutorial, a file already exists,
{\tt
\begin{verbatim}
   compressibility         uniform  150   300
   etaovers                uniform  0.05  0.32
   initial_flow            uniform  0.3   1.2
   initial_screening       uniform  0.0   1.0
   quenching_length        uniform  0.5   2.0
   initial_epsilon         uniform  15.0  30.0
\end{verbatim}
}
This implies that the model has four parameters. The names, without much inspiration, are {\tt par1}, {\tt par2}, {\tt par3} and {\tt par4}. These names would normally be more descriptive, e.g. {\tt NuclearCompressibility}. The second entry in each line is either {\tt uniform} or {\tt gaussian}. If the parameter is {\tt uniform}, the last two numbers represent the range of the uniform prior, $x_{\rm min}$ and $x_{\rm max}$. If the second entry is {\tt gaussian} the third entry represents the center of the Gaussian distribution and the fourth represents the width. For a real model, the User would replace this model with one appropriate for their own model.

The second file is {\tt \$\{MY\_PROJECTS\}/Info/observable\_info.txt}. This describes output values from the model. In the template the file is
{\tt
\begin{verbatim}
   meanpt_pion    100      
   meanpt_kaon    200      
   meanpt_proton  300      
   Rinv           1.0      
   v2             0.2     
   RAA            0.5
\end{verbatim}
}
The first entry in each line simply provides the names of the observable which will be processed in the Bayesian analysis.  The second entry is used by {\tt Smooth Emulator} during tuning, but only if a Monte Carlo method is used, and then is only used to seed the Monte Carlo search. If the analytical method is used for tuning (which is recommended) this parameter is irrelevant.

\subsection{Running {\it Simplex Sampler}}

Both {\it Simplex Sampler} and {\it Smooth Emulator} have options. These are provided in parameter files. For this tutorial, the provided parameter file is {\tt \$\{MY\_PROJECTS\}/parameters/simplex\_parameters.txt}. The provided file is
{\tt
\begin{verbatim}
   #Simplex_LogFileName    simplexlog.txt # comment out to direct output to screen
   Simplex_TrainType       2              # Must be 1 or 2             
   Simplex_ModelRunDirName modelruns      # Directory with training pt. info
\end{verbatim}
}
Because the first line is commented, the output of {\it Simplex Sampler} will be to the screen. Otherwise it would go to the specified file. By setting {\tt Simplex\_TrainType=1}, the sampler will choose $n+1$ training points, where $n=4$ is the number of model parameters. Each point corresponds to the vertices of an $n+1$ dimensional simplex.  Finally, the parameter {\tt Simplex\_ModelRunDirName} is set to ``{\tt modelruns}''. This informs {\tt Simplex Sampler} to write the coordinates of each training point and the corresponding observables in the directory {\tt \$\{MY\_PROJECTS\}/rhic/modelruns/}. 

Now the user can run {\tt Simplex Sampler}, which must be run from the project directory. The only output is the number of training points.
 {\tt
\begin{verbatim}
   ${MY_PROJECTS}/rhic% ${MY_LOCAL}/bin/simplex
   NTrainingPts=28
\end{verbatim}
}
If one had set {\tt Simplex\_TrainType}=1, only seven training points would have been created. The programs writes information about the training points in the {\tt modelruns/} directory. Changing into that directory, there should now be 28 sub-directories, corresponding to the 28 training points: {\tt modelruns/run0}, {\tt modelruns/run1}, {\tt modelruns/run2}, {\tt modelruns/}$\cdots$. Each directory has one text file describing the training points. For example, the {\tt modelruns/run0/mod\_parameters.txt} file might be 
{\tt
\begin{verbatim}
   compressibility 190.282
   etaovers 0.14892
   initial_flow 0.664958
   initial_screening 0.426807
   quenching_length 1.16036
   initial_epsilon 21.7424
\end{verbatim}
}
This describes the six model parameters, which will serve as the input for the first full model run.  The next step will be to run the full model for the parameters in each directory. Thus for {\tt Simplex\_Traintype=1}, one would need 7 full-model runs, and for {\tt Simplex\_Traintype=2}, one would need to do 28 full-model runs. The corresponding observables will be written in the files {\tt modelruns/runI/obs.txt}

\subsection{Running the Fake Full Model}
Once the training points have been generated, the user will input a Real full model based on the given structure, tailored to address their specific problem. For the tutorial, a fake model is provided. It reads the model-parameter values in each {\tt modelruns/runI/mod\_parameters.txt} file and writes the corresponding observables in {\tt modelruns/runI/obs.txt}. The output should be as follows:
{\tt
\begin{verbatim}
   ${MY_PROJECTS}/rhic% ${MY_LOCAL}/bin/fakerhic
   NTraining Pts=28
   NPars=6
\end{verbatim}
}
The output simply verifies the number of model parameters and the number of training points created by simplex.

Inspecting the {\tt modelruns/run0/obs.txt} file,
{\tt
\begin{verbatim}
   meanpt_pion   418.821195  1.000000
   meanpt_kaon   715.592889  2.000000
   meanpt_proton 1079.482871 3.000000
   Rinv          5.004248    0.010000
   v2            0.178353    0.002000
   RAA           0.553416    0.005000
\end{verbatim}
}
The second entry of each line is the value of the specified observable for that specific training point. The last entry is the random uncertainty associated with the full model. This is only relevant if the model has random fluctuations, meaning the re-running the model at the same point might result in different output. For this tutorial, the emulator will not consider such fluctuations (there is an emulator parameter that can be set to either consider the randomness or ignore it), so the third entry on each line is superfluous.

\subsection{Running {\it Smooth Emulator}}
To tune the emulator, the User will run {\tt \$\{MY\_LOCAL\}/bin/SmoothEmulator\_tune} which should have been compiled in the directions above. The User needs to edit one additional file a this point, the parameter file that sets numerous options for {\it Smooth Emulator}. For the template used in this tutorial, that file is

{\tt
\begin{verbatim}
#SmoothEmulator_LogFileName smoothlog.txt # comment out for interactive running
  SmoothEmulator_LAMBDA 2.0 # smoothness parameter
  SmoothEmulator_MAXRANK 5
  SmoothEmulator_ConstrainA0 false
  SmoothEmulator_ModelRunDirName modelruns
  SmoothEmulator_TrainingPts 0-27
  SmoothEmulator_UsePCA   false
  SmoothEmulator_TuneExact true
 #
 # These are only used if you are using MCMC tuning rather than Exact method
  SmoothEmulator_TuneChooseMCMC false # set false if NPars<5
  SmoothEmulator_TuneChooseMCMCPerfect false #
  SmoothEmulator_MCMC_NASample 8  # No. of coefficient samples
  SmoothEmulator_MCStepSize 0.01
  SmoothEmulator_MCMC_CutoffA false # Used only if SigmaA constrained by SigmaA0
  SmoothEmulator_MCSigmaAStepSize 1.0  #
  SmoothEmulator_MCMCUseSigmaY false # If false, also varies SigmaA
  SmoothEmulator_MCMC_NMC 20000  # Steps between samples 
 #
 # This is for the MCMC search of parameter space (not for the emulator tuning)
 MCMC_METROPOLIS_STEPSIZE 0.01
\end{verbatim}
}
The parameters are described in detail in Sec. \ref{sec:emulator}. Because {\tt SmoothEmulator\_TuneExact} is set to {\tt true}, the Monte Carlo methods are not invoked and none of the parameters with {\tt MCMC} in their names are relevant. The most relevant parameter is setting the smoothness parameter. Also, it is important to make sure that {\tt SmoothEmulator\_TrainingPts} is set to the correct number of training points. The Constrain A0 parameter decides where the first term of the Taylor expansion is used to estimate the variance of the coefficients, which then affects the emulator's estimate of its uncertainty.

Now, running {\tt smoothy\_tune}, produces the following output,

{\tt
\begin{verbatim}
${MY_PROJECTS}/rhic% ${MY_LOCAL}/bin/smoothy_tune
 ---- Tuning for meanpt_pion ----
 ---- Tuning for meanpt_kaon ----
 ---- Tuning for meanpt_proton ----
 ---- Tuning for Rinv ----
 ---- Tuning for v2 ----
 ---- Tuning for RAA ----
\end{verbatim}
}
The program generates Taylor coefficients which are saved in the {\tt coefficients/} directory. Each observable has its own sub-directory with its name. In this case, {\tt smoothy\_tune} created the directories, {\tt coefficients/rhic/RAA}, {\tt coefficients/Rinv}, {\tt coefficients/menapt\_kaon}, {\tt coefficients/meanpt\_pion}, {\tt coefficients/meanpt\_proton} and {\tt coefficients/v2}. Within each of these sub-directories {\tt smoothy\_tune} created files {\tt meta.txt}, {\tt ABest.txt} and {\tt BetaBest.txt}.The number or parameters, the maximum rank of the Taylor expansion and the overall number of Taylor coefficients are give in {\tt meta.txt}. The file {\tt ABest.txt} provides the actual coefficients of the Taylor expansion, and {\tt BetaBest.txt} gives an array used to calculate the uncertainty. If one of the Monte Carlo methods is chosen, rather than the default analytic tuning method, the file {BetaBest.txt} is replaced by several files, {\tt sample0.txt, sample1.txt}$\cdots$, which provide several samples of Taylor coefficients. For the tutorial, the parameter file {\tt parameters/emulator\_parameters.txt} has the parameters set to use apply analytic tuning rather than Monte Carlo tuning.

\section{Testing the Emulator at the Training Points}
{\it Smooth Emulator} should return the training values at the training points. If one runs the executable {\tt smoothy\_train\_test}, it will first read in the coefficient information along with the training information. The program then emulates the model at the training points and compares the emulated value to the training value. Running the program gives the output:
{\tt
\begin{verbatim}
${MY_PROJECTS}/rhic% ${MY_LOCAL}/bin/smoothy_train_test
 --- TESTING AT TRAINING POINTS ----
 ------ itrain=0 --------
 Y[0]= 4.188e+02 =?  4.188e+02,    SigmaY_emulator= 1.78365e-07
 Y[1]= 7.156e+02 =?  7.156e+02,    SigmaY_emulator= 2.81059e-07
 Y[2]= 1.079e+03 =?  1.079e+03,    SigmaY_emulator= 4.08783e-07
 Y[3]= 5.004e+00 =?  5.004e+00,    SigmaY_emulator= 3.15227e-09
 Y[4]= 1.784e-01 =?  1.784e-01,    SigmaY_emulator= 1.08732e-10
 Y[5]= 5.534e-01 =?  5.534e-01,    SigmaY_emulator= 4.26570e-10
 ------ itrain=1 --------
 Y[0]= 4.744e+02 =?  4.744e+02,    SigmaY_emulator= 1.82174e-07
 Y[1]= 7.156e+02 =?  7.156e+02,    SigmaY_emulator= 2.87061e-07
 Y[2]= 1.066e+03 =?  1.066e+03,    SigmaY_emulator= 4.17513e-07
 Y[3]= 5.004e+00 =?  5.004e+00,    SigmaY_emulator= 3.21959e-09
 Y[4]= 1.784e-01 =?  1.784e-01,    SigmaY_emulator= 1.11054e-10
 Y[5]= 5.533e-01 =?  5.533e-01,    SigmaY_emulator= 4.35679e-10
 ------ itrain=2 --------
 Y[0]= 4.437e+02 =?  4.437e+02,    SigmaY_emulator= 3.01087e-07
 Y[1]= 7.846e+02 =?  7.846e+02,    SigmaY_emulator= 4.74437e-07
 Y[2]= 1.073e+03 =?  1.073e+03,    SigmaY_emulator= 6.90041e-07
 Y[3]= 5.004e+00 =?  5.004e+00,    SigmaY_emulator= 5.32114e-09
 Y[4]= 1.784e-01 =?  1.784e-01,    SigmaY_emulator= 1.83543e-10
 Y[5]= 6.175e-01 =?  6.175e-01,    SigmaY_emulator= 7.20065e-10
 ------ itrain=3 --------
 Y[0]= 4.457e+02 =?  4.457e+02,    SigmaY_emulator= 2.47694e-07
 Y[1]= 6.842e+02 =?  6.842e+02,    SigmaY_emulator= 3.90304e-07
 Y[2]= 1.182e+03 =?  1.182e+03,    SigmaY_emulator= 5.67674e-07
\end{verbatim}
$\vdots$
}
The observables, $Y[0]\cdots Y[27]$ should be identical and the uncertainties at the training points should be zero. The fact that the uncertainties are not exactly zero derives from the numerical accuracy of the linear algebra routines.

\section{Generating Emulated Observables at Given Points}
Finally, now that the emulator is tuned, one may wish to generate emulated values for the observables for specified points in model-parameter space. A sample program, {\tt \$\{MY\_LOCAL\}/bin/smoothy\_calcobs} is provided to illustrate how this can be accomplished. If one invokes the executable, using the same parameters as those used by {\tt smoothy\_tune}, the User is prompted to enter the coordinates of a point in model-parameter space, after which {\tt smoothy\_calcobs} prints out the observables. In this case, for the case where {\tt compressibility=205}, {\tt etaovers=0.2}, {\tt initial\_flow=0.7}, {\tt initial\_screening=0.4}, {\tt quenching\_length=1.2} and{\tt initial\_epsilon=23.0}

{\tt
\begin{verbatim}
${MY_PROJECTS}/rhic% ${MY_LOCAL}/bin/smoothy_calcobs
 Prior Info
 #   ParameterName Type   Xmin_or_Xbar  Xmax_or_SigmaX
  0: compressibility     uniform        150        300
  1: etaovers            uniform        0.05       0.32
  2: initial_flow        uniform        0.3        1.2
  3: initial_screening   uniform          0          1
  4: quenching_length    uniform        0.5          2
  5: initial_epsilon     uniform         15         30
 Enter value for compressibility:
 205
 Enter value for etaovers:
 .2 
 Enter value for initial_flow:
 .7
 Enter value for initial_screening:
 0.4
 Enter value for quenching_length:
 1.2
 Enter value for initial_epsilon:
 23.0
 ---- EMULATED OBSERVABLES ------
 meanpt_pion = 425.843 +/- 2.15299
 meanpt_kaon = 747.019 +/- 3.39257
 meanpt_proton = 1084.91 +/- 4.93428
 Rinv = 5.17076 +/- 0.03805
 v2 = 0.181905 +/- 0.00131247
 RAA = 0.59458 +/- 0.00514898
\end{verbatim}
}
Note that the uncertainties for the emulation are not effectively zero, as each set of the 8 sets of coefficients provides an an emulator that exactly reproduces the training points.

Of course, it is unlikely the User will wish to enter model parameters interactively as was done above. To incorporate {\tt Smooth Emulator} into other programs, the User should inspect the main programs, e.g. {\tt \$\{MY\_LOCAL\}/main\_programs/smoothy\_calcobs\_main.cc}. The User can then design their own program based on this source code, and compile and link it by editing {\tt \$\{MY\_LOCAL\}/main\_programs/CMakeLists.txt}. By editing the CMake file, replacing the lines unique to {\tt smoothy\_calcobs}, one can easily compile new executables based on the User's main programs. To understand what might be involved, the source code in {\tt \$\{MY\_LOCAL\}/main\_programs/SmoothEmulator\_calcobs\_main.cc} is
{\tt
\begin{verbatim}
#include "msu_smoothutils/parametermap.h"
#include "msu_smooth/master.h"
#include "msu_smoothutils/log.h"
using namespace std;
int main(){
   NMSUUtils::CparameterMap *parmap=new CparameterMap();
   parmap->ReadParsFromFile("parameters/emulator_parameters.txt");
   NBandSmooth::CSmoothMaster master(parmap);
   master.ReadCoefficientsAllY();
   NBandSmooth::CModelParameters *modpars=new NBandSmooth::CModelParameters(); // contains info about single point
   modpars->priorinfo=master.priorinfo;
   master.priorinfo->PrintInfo();
   
   // Prompt user for model parameter values
   vector<double> X(modpars->NModelPars);
   for(unsigned int ipar=0;ipar<modpars->NModelPars;ipar++){
      cout << "Enter value for " << master.priorinfo->GetName(ipar) << ":\n";
      cin >> X[ipar];
   }
   modpars->SetX(X);
   
   //  Calc Observables
   NBandSmooth::CObservableInfo *obsinfo=master.observableinfo;
   vector<double> Y(obsinfo->NObservables);
   vector<double> SigmaY(obsinfo->NObservables);
   master.CalcAllY(modpars,Y,SigmaY);
   cout << "---- EMULATED OBSERVABLES ------\n";
   for(unsigned int iY=0;iY<obsinfo->NObservables;iY++){
      cout << obsinfo->GetName(iY) << " = " << Y[iY] << " +/- " << SigmaY[iY] << endl;
   }

   return 0;
}
\end{verbatim}
}
This illustrates how one can write a code that 
\begin{itemize}\itemsep=0pt
\item[a)] Reads the parameter file
\item[b)] Creates a {\it master} emulator file (called master because it includes an emulator for each observable)
\item[c)] Creates a model-parameters object, {\tt modpars}, that stores the coordinates of the model-parameter point
\item[d)] Calculates the observables from the emulator
\end{itemize}


\end{document}
