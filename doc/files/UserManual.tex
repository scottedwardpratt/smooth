\documentclass[12pt]{article}
%#Scott Pratt, Eren Erdogan, Ekaksh Kataria 2023
\usepackage{subfiles}
%\usepackage{indentfirst}
\usepackage{graphicx}
\usepackage{enumitem}
\usepackage{float}
\usepackage[
        pdfencoding=auto,%
        pdftitle={Smooth Emulator and Training Point Optimizer}
        pdfauthor={Scott Pratt},%
        pdfstartview=FitV,%
        colorlinks=true,%
        linkcolor=blue,%
        citecolor=blue, %
        urlcolor=blue,
				breaklinks=true]{hyperref}
%\usepackage[anythingbreaks,hyphenbreaks]{breakurl}
\usepackage{xurl,hyperref}
\usepackage[utf8]{inputenc}
\usepackage{comment}
%\usepackage{pdfsync}
\usepackage{amssymb}
\usepackage{amsmath}
\usepackage{color}
%\usepackage{nopageno}
\usepackage{bm}
\usepackage{dsfont}
%\usepackage[utf8]{inputenc}
\usepackage[small,bf]{caption}
%\usepackage{fontspec}
%\usepackage{textcomp}
%\usepackage{color}
%\usepackage{fancyhdr}
\usepackage[titletoc]{appendix}
%\usepackage[headheight=110pt]{geometry}
\usepackage{bm}

\def\theenumi{\Roman{enumi}}

\numberwithin{equation}{section} 
\numberwithin{figure}{section}

\usepackage[most]{tcolorbox}
\tcbset{
frame code={}
center title,
left=0pt,
right=0pt,
top=0pt,
bottom=0pt,
colback=gray!25,
colframe=white,
width=\dimexpr\textwidth\relax,
enlarge left by=0mm,
boxsep=5pt,
arc=0pt,outer arc=0pt,
}

\usepackage{xcolor}
\definecolor{darkred}{rgb}{0.85,0.0,0.0}
\definecolor{lightgray}{rgb}{0.85,0.85,0.85}
\newenvironment{sloppypar*}
 {\sloppy\ignorespaces}
 {\par}

%\usepackage{fancyvrb}[commandchars=\\\{\}]
\usepackage{fvextra}[commandchars=\\\{\}]
%\DefineVerbatimEnvironment{Highlighting}{Verbatim}{
%commandchars=\\\{\},
%breaklines, breaknonspaceingroup, breakanywhere}
\fvset{breaklines}
\fvset{highlightlines=1-1000}
\fvset{highlightcolor=lightgray}

%\usepackage[most]{tcolorbox}
%\tcbset{
%frame code={}
%center title,
%left=0pt,
%right=0pt,
%top=0pt,
%bottom=0pt,
%colback=gray!25,
%colframe=white,w
%width=\dimexpr\textwidth\relax,
%enlarge left by=0mm,
%boxsep=5pt,
%arc=0pt,outer arc=0pt,
%}
%\newcounter{examplecounter}
%\counterwithin{examplecounter}{section}
%\setcounter{examplecounter}{0}
%\newcommand{\example}[2]{\begin{tcolorbox}[breakable,enhanced]
%\refstepcounter{examplecounter}{
%\bf Example \arabic{section}.\arabic{examplecounter}:}~~{\bf #1}\\
%{#2}
%\end{tcolorbox}
%}

%\newcommand{\exampleend}{
%\begin{samepage}
%\nopagebreak\noindent\rule{\textwidth}{1pt}
%\end{samepage}
%}

%\usepackage{silence}
%\WarningFilter{hyperref}{Token not allowed in a PDF String}

\newcommand\eqnumber{\addtocounter{equation}{1}\tag{\theequation}}
%\newcommand{\solution}[1]{ }
\newcommand\identity{\mathds{1}}

\setcounter{tocdepth}{2}

\setlength{\headheight}{16pt}
\parskip 6pt
\parindent 0pt
\textwidth 7.0in
\hoffset -0.8in
\textheight 9.2in
\voffset -1in

%\newcommand{\bm}{\boldmath}
%\boldmath
%

\begin{document}

\begin{titlepage}
   \begin{center}
       \vspace*{1.5cm}

       {\bf\LARGE {\it Smooth Emulator \&  Training Point Optimizer}}\\
       \vspace*{8pt}

       {\bf\Large User Manual}

       {\bf A BAND Collaboration Project}\\
       \href{https://bandframework.github.io}{https://bandframework.github.io}
       
       \begin{center}
     
       \includegraphics[width=0.24\textwidth]{figs/BAND_logo.png}
      \end{center}
      


       \vspace{0.25cm}

       {\large Scott Pratt, Oleh Savchuk, Eren Erdogan, Ekaksh Kataria}

       {\it Department of Physics and Facility for Rare Isotope Beams}

       {\it Michigan State University, East Lansing Michigan, 48824}

       \today
  \end{center}
       
    \vspace*{3.0cm}
     

     \hspace*{2.0cm}
      \includegraphics[width=0.16\textwidth]{figs/FRIB_logo.png}

       \vspace*{-3cm}
       \hspace*{6.35cm}
       \includegraphics[width=0.26\textwidth]{figs/msu_logo}

       
       
       \vspace*{-1.75cm}
       \hspace*{12.5cm}
       \includegraphics[width=0.12\textwidth]{figs/nsf_logo.png}

       
%\vspace*{0.5cm}

\begin{center}
{\it This project was supported by the National Science Foundation\\ through the Office of Cyberinfrastructure for Sustained Scientific Motivation (CSSI)}
\end{center}

\end{titlepage}

\newpage

\thispagestyle{empty}

\tableofcontents

\newpage

\thispagestyle{empty}

\section{Overview}\label{sec:overview}

{\it Smooth Emulator} software is designed with the idea that the full-model to be emulated is {\it smooth}. I.e., if one were represent the behavior as a Taylor expansion, one would expect higher-order terms to contribute less than lower-order terms. The programs that choose the training points are based on such an assumption, and the emulator's mathematical form also presumes such behavior. Applying the software to emulation of a full-model that has sharp peaks or valleys is probably dubious. The programs tend to run quickly as long as the number of model-parameters is of the order a dozen or less.

The software performs three basic functions.
\begin{enumerate}\itemsep=0pt
\item Provide an optimum set of training points in model parameter space, at which full model runs will performed to then tune the emulator. The User must first provide a list of model parameters and their priors, along with a measure of the relative importance of each model parameter.
\item After the User provides a list of observables, the User runs their full model at each of the training points and provides values and uncertainties for each observable at each training point. {\it Smooth Emulator} software then creates and tunes emulators for each observable. The emulator form is based on two hyper-parameters, an amplitude and a convergence length. Both hyper-parameters are found by the tuning procedure. The emulator is mathematically equivalent to a Gaussian process emulator with a quadratic kernel.
\item Once the User provides a list of experimentally measured observables and uncertainties, an MCMC program samples the prior and provides a sampling of points in the model-parameter space that are consistent with the posterior likelihood. An analysis program uses the list of representative points to calculate the mean and covariances of the posterior distribution. The program also provides the resolving power of how each observable constrains each model parameter.
\end{enumerate}

The structure is designed so that the User stores all the information for a specific analysis in a specific analysis directory, e.g. {\tt  \$\{MY\_ANALYSIS\}/}. All programs, aside from PYTHON scripts for plotting, are then run from that directory.  A subdirectory, {\tt  \$\{MY\_ANALYSIS\}/smooth\_data/Info}, contains simple text files edited by the User to describe the model-parameter priors, the observables to be emulated and evaluated, and the experimentally observed values. A second subdirectory, {\tt \$\{MY\_ANALYSIS\}/smooth\_data/Options} contains text files defining options the User can set related to the {\it Smooth Emulator} software. The training point optimization software creates directories {\tt  \$\{MY\_ANALYSIS\}/smooth\_data/FullModelRuns/runI} where each training point, {\tt $\hdots$/runI}, refers to a single point in model-parameter space used for training. The training point optimizer program creates a listing of a set of training points, where the values of the model parameters for each training point are written to files. The User's job is then to create text files giving the values of the observables calculated by the full model for each training point. These files are stored in the directory {\tt \$\{MY\_ANALYSIS\}/smooth\_data/FullModelRuns/}. The MCMC program writes the MCMC trace information to files in {\tt smooth\_data/MCMC/}. {\it Smooth Emulator} provides programs that analyze the trace to provide averages and covariances of the the posterior, as well as measures of the resolving power. Several PYTHON scripts are located in {\tt \$\{MY\_ANALYSIS\}/figs/}.  

{\it Smooth Emulator} software is designed to be modular. Programs for each of the three functionalities listed above can be performed independently because files are all in simple text formats. For example, one can generate the training points, then use those training points emulators from other software packages. However, the User will likely have to spend more time translating text files to formats used by other programs. Aside from the plotting scripts, {\tt Smooth Emulator} software is written in C++, the software, and this manual, were built on the assumption that the User has some familiarity with C++.

\newpage

\subfile{installation.tex}

\newpage

\subfile{theory.tex}

\newpage

\subfile{tpo.tex}

\newpage

\subfile{fullmodel.tex}

\newpage

\subfile{emulator.tex}

\newpage

\subfile{mcmc.tex}

\newpage

\subfile{figs.tex}

\newpage

\subfile{parametermap.tex}

\newpage

\subfile{tutorial.tex}

\end{document}

